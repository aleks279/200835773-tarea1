\documentclass{IEEEtran}
\usepackage{graphicx}
\usepackage{fancyhdr}
\usepackage{listings}

\graphicspath{ {images/} }
\pagestyle{fancy}
\fancyhf{}
\rhead{Tarea \#1}
\rfoot{P\'agina \thepage}

\begin{document}
\begin{titlepage}
  \centering
  {\scshape\LARGE Instituto Tecnol\'ogico de Costa Rica \par}
  \vspace{1cm}
  {\scshape\Large Tarea \#1\par}
  \vspace{1.5cm}
  {\Large\itshape Sa\'ul Zamora\par}
  \vfill
  profesor\par
  Kevin Moraga \textsc{}

  \vfill

% Bottom of the page
  {\large \today\par}
\end{titlepage}

\section{Introducci\'on}
El objetivo principal de esta tarea es la introducci\'on al an\'alisis de bases de datos ya existentes. Para lo cual se lleva a cabo un desarrollo dividido en dos partes.

\subsection{Parte 1}
En la primera parte, se desarrolla un modelo de base de datos relacional para una base de datos encargada de organizar duelos del conocido juego de cartas \emph{Magic The Gathering}.

\subsection{Parte 2}
La segunda parte consiste en el an\'alisis de una base de datos existente. Utilizando la base de datos p\'ublica prove\'ida por el profesor, '

\section{Ambiente de desarrollo}
Para el realizar el modelo de base de datos entidad-relaci\'on de \emph{Magic The Gathering}, se utilizo la herramienta \emph{StarUML} para generar el diagrama UML. Dicha herramienta tambi\'en se utiliz\'o para realizar el diagrama de base de datos entidad-relaci\'on de la base de datos compartida.'

\section{Estructuras de datos usadas y funciones}

\section{Instrucciones de ejecuci\'on'}

\section{Bit\'acora de trabajo}
\begin{itemize}
  \item 25-02-2017:
  \begin{itemize}
    \item 1.5 horas - modelado de base de datos de Magic (texto).
    \item 2 horas   - modelado de base de datos de Magic (StarUML).
    \item 0.5 horas - documentaci\'on.
  \end{itemize}
\end{itemize}

\section{Comentarios finales}

\section{Conclusiones}

\begin{itemize}
  \item 
\end{itemize}

\begin{thebibliography}{99}
 \bibitem{magic} Cómo jugar magic the gathering. Retrieved February 25, 2017, from  \texttt{http://es.wikihow.com/jugar-Magic-The-Gathering}
\end{thebibliography}
\end{document}